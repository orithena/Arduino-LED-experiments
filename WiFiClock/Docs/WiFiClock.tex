\documentclass[a4paper,10pt,twocolumn]{article}

%\usepackage{xltxtra}
\usepackage{longtable}
\usepackage[utf8]{inputenc}

%\setromanfont[Mapping=tex-text]{Linux Libertine O}
%\setsansfont[Mapping=tex-text]{DejaVu Sans}
% \setmonofont[Mapping=tex-text]{DejaVu Sans Mono}
\columnsep3em

\title{WiFi-LED-Clock}
\author{Dave (dave@kliczbor.de)}
\date{2018-03-02}

\begin{document}
\maketitle
 
\section{Description}
You have in your hands one of the most unique clocks on the market today. With it's
classic, ticking, analogue clockwork and futuristic, silent, colorful, digital Light 
Emitting Diodes it surely would give your kitchen an eyecatcher people will talk about.

Don't use it in your bedroom though, its ticking is too loud and the lights will 
interrupt a good nights sleep. It could, however, double as a nightlight in the corridor
between your bedroom and the bathroom.

We removed the need for batteries from this cheap IKEA clock. Which, by the way, isn't sold
anymore by IKEA, but old stock is already sold at over ten times the old price on the market.
You see, you've got a real collectors item here.

This clock comes with a 5 volt power supply to drive both the LEDs and the clock itself.
It needs to connect to your WiFi network to get the time from a time server, so that the light
patterns change according to the time of day. Some modes even dim the light at night. It is
important to note here that the analog clockwork is not set digitally, you will need to 
do that manually. Also, the digital part is programmed to follow the timezone ``Europe/Berlin''
according to the rules active in 2017. If germany ever moves to abolish daylight savings time,
this clock needs to be reprogrammed.

\section{Setting it up}

The power supply goes into a wall socket. Duh.

After the clock is powered on, it checks whether the stored 
WiFi credentials are still valid. As long as WiFi is not connected, 
the clock shows a solid red.

If this is not the case, it goes into access point mode, opening a WiFi with the name 
``ClockSetup''. Use your smartphone to connect to it. You should be presented with a
dialog to connect to the network, like when you try to access a public WiFi and first have
to accept the Allgemeine Geschäftsbedingungen. But there is no legal mumbo-jumbo to accept; 
instead, you could scan for your WiFi network and enter the pre-shared key (also known 
as ``PSK'' or ``WiFi password'').

If this for some reason does not work, try to set up the WiFi connection on your phone or laptop using the following
data:

\begin{tabular}[h]{ll}
SSID & ClockSetup \\
Static IPv4 address & 192.168.4.4 \\
Subnet mask & 255.255.255.0 \\
Gateway & 192.168.4.1 
\end{tabular}

Then, point your web browser at http://192.168.4.1/, enter the credentials 
of your WiFi network and you're good to go.

Note that the clock will open that setup portal iff

\begin{enumerate}
 \item it is powered on (or the chip is resetted)
 \item and it cannot connect to the WiFi network previously configured.
\end{enumerate}

As soon as the WiFi is successfully connected, the clock shows a solid blue. When it
successfully resolved the IP address of a time server, it displays a solid green until
the time server answers with the current time. Then the clock enters the default mode
(see section \ref{mode}).

You also will need to set the clock hands position using the dial on the backside 
of the clockwork. It is not advisable to try and synchronize the seconds hand with the 
digital LED display, because they will likely to be out of whack after a few weeks \dots{} and
anyway, the seconds are hardly noticeable in the LED display, unless you know that red
corresponds to zero in the HSV color model.

But if you achieve that, we applaud you!

\section{Hang it up!}
\label{hangup}
We included a hook and two nails (in case you hit the first nail wrong) to hang the clock up 
on your wall. We trust you to find a good spot for it and we also trust that you check the
wall for hidden cables and pipes before not hitting yourself on your own fingers with a 
hammer. Which, by the way, you have to provide. Because we don't. Sorry.

You might notice that it might wobble a bit on the wall. This is partly intentional. You 
see, there is a button on the backside to switch modes and its position is determined in
a way that you are able to press the front of the clock to push said button against the wall
and thereby switching modes. 

However, you may turn the fake battery the button is 
resting on by a few degrees to let the button sink more into the case. This might mean that 
you'd have to take the clock off the wall to switch modes. Your choice.

\section{Mode setting}
\label{mode}

As we already told you in section \ref{hangup} how to press the mode button, you might be 
curious about its function. You see, no customer is alike, and we strive to accomodate a
wide range of tastes. But, as we like rainbows, all modes are somehow connected to rainbows.
Sorry if that does not fit your taste. Try mode 10 or 11, they feature the least noticeable 
rainbow.

Okay, by pressing the button, you choose the next mode. One of the twelve LEDs lights up white
for a moment to show you which mode is currently active.

After powering on, mode 0 is active. Sorry, I mean mode 12. It's mode 12 because of the 
12-hour clock system. And simultaneously 0, because of the 24-hour system. 
Does that even make sense what I'm talking here? Anyway, it's 12, because the top center 
LED is assigned to it when choosing modes. Also, it's 0, because the LEDs are controlled
internally using a modulo 12 ring.

\begin{description}%
\setlength\itemsep{0pt}%
\item[0/12] Medium bright rainbow ticking with the seconds around the clock, hour and minute hands highlighted with red 
	and blue, brightness dimming at night.
\item[1]    Medium bright rainbow ticking around the clock, hour and minute hands highlighted with red 
	and blue, constant brightness.
\item[2]    Constantly full bright rainbow ticking around the clock, hour and minute hands falling into
        shadow.
\item[3]    A bright rainbow ticking around the clock, dimmed at night.
\item[4]    A constantly medium bright rainbow ticking around the clock.
\item[5]    A constantly full bright rainbow ticking around the clock.
\item[6]    A bright rainbow ticking fast around the clock, dimmed at night.
\item[7]    Medium bright rainbow ticking fast around the clock, hour and minute hands highlighted with red 
	and blue, constant brightness.
\item[8]    A constantly full bright rainbow ticking around the clock, but slowly with the minutes.
\item[9]    A bright rainbow ticking around the clock, but slowly with the minutes, dimmed at night.
\item[10]   Solid color cycling through the rainbow, changing with every second, dimmed at night.
\item[11]   Solid color cycling through the rainbow, changing with every second, full brightness.
\end{description}


\section{Changing the background}

The background is just a paper circle with a diameter of 185 millimeters. Use the 
graphics program of your choice to design it. We recommend Inkscape, although you likely will
need to google the method to distribute 12 objects evenly spaced around a circle. 

We think that a grayscale design looks better under the colored lights, but if you
want a design that plays with being bathed in light of changing color, go for it!
This can lead to some very interesting effects!

Don't forget to mark a center and a cutting perimeter in your drawing.

To swap it, you need to first power off the clock. Then you need to remove the front 
pane of the clock. Locate the three nubs on the 
backside of the clock poking out of the perimeter and push them in using e.g. a flat screwdriver.

Be very careful not to damage the copper wires between the LEDs in the process! 

Use the same screwdriver to push the pane further out the front, then pull the pane off from 
the front using your fingers.

Remove the three clock hands by tentatively pulling them off. Change the background paper.
Don't put the new one on top of the old one, this might be getting too thick and/or wavy
and the hour hand should never scrape on the paper!

If in doubt, use some bits of double sided tape to fix the new background to the case.

Reattach the three clock hands by putting them all in the 12 o'clock position, else the 
hours hand might seem to be off after you adjust the clock. Squeeze the front pane in so 
that the nubs align to their slot in the case, then push the whole front pane back into 
its place.

Double check that the paper does not billow up by the pressure of the pane on its perimeter
and hinder the natural way of the hours hand. If in doubt, remove the front pane again and
use more double-sided tape \dots{} or find a way to push the pane back in without the paper
billowing up. The latter is not easy, but in most cases possible. That much we can tell 
you from our own experience.

\section{Firmware upgrade}
If, for some reason, a firmware upgrade is needed, it is administered through the micro 
USB port on the back side of the clock. 

Right next to that port, by the way, is a small
hole to access the reset button, which you will not need unless you choose to hack around
with the clocks firmware yourself. Do ask for our source code, we will happily provide it
along with some pointers on how to get started using the Arduino IDE.

Here's the basic data: The chip is an ESP8266 on a WeMos D1 mini board with a CH340G
USB-to-Serial converter to program it. The LEDs are 
of the type ``WS2812b'' in GRB color byte order, connected to pin D4. LED 0 is on the 
clocks position 11, counting anticlockwise; the function to calculate from the intuitive
position number (a.k.a. hours) to the actual led number is 
$$lednumber(h) = 11 - (h \bmod 12)$$
with $h$ being the ``intuitive'' hour position.

The mode button is active low, connected to pin D3 with an internal pull-up resistor.

\section{Summary}
Are you happy with your new clock? Then tell your friends that they have no chance of getting 
the same clock, unless they somehow manage to lay their hands on such a fine collectors 
item named RIBBA.

Even then, they'll have to ask us for the source code and shopping list --- and they'll need a 
steady hand to solder it.

\end{document}
